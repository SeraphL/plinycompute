
\begin{center}
\textbf{Postdoctoral Mentoring Plan}
\end{center}

The presence of a postdoctoral researcher is essential in this project
because of its broad scope. This person will ideally have previous
training in systems and/or database research, 
and will be
able to grasp from the very beginning the broader goals of the project
and help the Ph.D. students who will be recruited for this
project. 
He or she will work closely with PI Jermaine.
This person will also gain an
understanding of issues related to the collaborations with graduate students.
and dissemination of software as discussed under the Broader Impacts.
Specifics of the training program include the following:

\vspace{-5pt}\paragraph{Orientation}
Discuss the postdoc's expectations of the mentor/mentee relationship, the source of funding for the postdoc's compensation, and the scientific and educational goals of the 
project.  
\vspace{-5pt}\paragraph{Building trust}
Ask the postdoc about his/her goals and respect and accept them where
these are reasonable; if not reasonable, attempt to steer the postdoc to
more realistic goals.  
Encourage feedback from the postdoc regarding his/her need for guidance.
Watch for depression, fatigue, isolation.

\vspace{-5pt}\paragraph{Education/training}
Explicitly instruct the postdoc in research methodology and techniques,
effective problem-solving strategies, and critical interpretation of the 
scientific literature.  Inform the postdoc of available resources and be 
willing to refer him/her to someone else for help/information.
Discuss a timeline for progress in the research project.
Encourage the postdoc to seek additional mentors in the various areas
in which s/he will require guidance.
Involve the postdoc in establishing successful collaborations.
Promote ethical standards for conducting research including compliance
with all institutional and federal regulations as they relate to responsible
conduct in research.
% Define expectations for the ethical conduct of
% research.
% Discuss the ground rules of collaboration; clarify collaborative
% issues such as ownership and sharing of research results and proper
% attribution of contributions to the research.  
% Discuss conflict of interest, grant applications, and ethical
% implications of the postdoc's research.
% Involve the postdoc in scientific discussions within group meetings
% and/or on an individual basis.
Provide opportunities for the postdoc to participate in the writing
and reviewing of papers and grant applications.  
% Encourage the postdoc to take the time to attend meetings, seminars
% and other career development activities, maintaining an appropriate balance
% between these activities and the development of the postdoc's research 
% career.
% Establish and communicate rules of authorship.
% Ensure that the research performed by a postdoc is submitted for
% publication in a timely manner and that s/he receives appropriate credit for
% the work performed.  
% Encourage creativity and independence.
\vspace{-3pt}
\paragraph{Evaluation}
Conduct periodic reviews with the postdoc.
% Assess the postdoc's progress to date, strengths, areas needing
% improvement, and potential for a research career in the discipline.
Maintain open communication with the postdoc regarding career goals
and options. %Include a periodic review of mutual expectations.
Offer candid assessment of the postdoc's potential to become
an independent investigator.
\vspace{-3pt}

\paragraph{Career preparation}
%Support/encourage the postdoc to present their work at scientific
%meetings. 
Help the postdoc engage in networking; introduce him/her to
colleagues at meetings or by phone/email.
Play an active role in the postdoc's job search (provide advice on
applications, CVs, interviews, presentations, negotiations, etc.).
Offer opportunities for the postdoc to develop supervisory skills
through training students and other research staff.
% Encourage the postdoc to participate in career development seminars
% and activities offered by Rice University.
Encourage the postdoc to actively seek opportunities for professional
experience and advancement (e.g., volunteer on committees, organize
scientific meetings/retreats).
% Maintain a positive attitude toward a diverse range of career
% opportunities for the postdoc; learn about nonacademic opportunities and
% provide appropriate advice for postdocs interested in nontraditional career
% paths.
In addition to the mentoring provided within the proposed project,  
Rice University offers exceptional opportunities for the training of the
postdoctoral researcher. These include:
\begin{itemize}
\item Participation in NSF ADVANCE activities which include lunches on issues that concern postdoctoral students and an annual very successful workshop ``Negotiating the Ideal Faculty Position.''
\item Rice's Grant writing workshop for NSF.
\item Rice's training in using large clusters and facilities available
  at National Labs.
\item Participation in postdoc lunches organized by the Dean of Graduate and Postdoctoral Studies at Rice that cover topics relevant to this community.
\item Participation in the job market workshop, conflict resolution workshop and the entrepreneurship club, all organized by the Dean of Graduate and Postdoctoral Studies.
\end{itemize}
