\section{Object Model Tuning}

\noindent
\textbf{This section is just a preliminary draft.}

\vspace{5 pt}
\noindent
The PC \texttt{Object} model is designed for zero-cost data movement, the result being that there is often no (de-)serialization cost
for moving PC \texttt{Object}'s accross processes.  But memory management can still be costly.  Deallocating and cleaning
up complex objects (in particular, instances of container classes) can require significant CPU resources, which, depending upon the 
circumstance, may be un-necessary.  In-keeping with the assertion that application programmers shold be in
control of performance-critical policies, it is possible to explicitly control how memory is reclaimed and re-used during computations.

\subsection{Allocator Policies}

When the reference count for a PC \texttt{Object} located in a managed allocation block goes to zero, it is deallocated.  The exact
meaning of ``deallocated'' is controllable by the programmer, via a call to \texttt{setAllocatorPolicy}.  Currently, PC ships with
three (de)allocator policies:

\begin{enumerate}

\item Lightweight re-use.  This is the default policly.  When an \texttt{Object} is deallocated, its space in the allocation block is made available for re-use by
adding the space to a pool of similarly-sized, recycled memory chunks (specifically, all recycled chunks are organized into buckets, where a chunk of size
$n$ goes into bucket $\log_2 (n)$).  A request for RAM in a block is fulfilled by first scanning the recycled chunks in the appropriate bucket, then
attempting to allcate new space on the end of the page, if that fails.
\item No re-use.  The space containing deallocated \texttt{Object}s is not-reused.  Hence, it is very similar to classical, region-based allocation---though \texttt{Object}s
are reference-counted, and a destructor is called for each unreachable \texttt{Object}.
On the positive side, this allocation policy is very fast.  On the negative side, frequent allocations of temporary \texttt{Object}s will result in a lot of wasted space.
\item Full re-use.  This policy uses a heavy-duty allocator, similar to what one would expect would ship with \texttt{malloc}.  Compared to lightweight
re-use, full re-use will attempt to combine adjacent unused chunks into larger chunks, and will try to use the best fit among unused chunks to service
allocation requests.

\end{enumerate}

It is also possible for a programmer to supply the following policies, on a per-\texttt{Object} bases, during \texttt{Object} allocation:

\begin{enumerate}

\item No reference counting.  This \texttt{Object} is not reference counted, and it is not included in the total count of \texttt{Object}s on an allocation
block.  If each \texttt{Object} on an allocation block
is allocated in this way, this results in pure, region-based memory management, and is exceedingly lightweight.
\item Full reference counting.  This is the default.
\item Unique ownership.  The \texttt{Object} is not reference counted, but there can only be one \texttt{Handle} object referencing the uniquely-owned
object at a time.  When that \texttt{Handle} is destroyed, the objec is deallocated.

\end{enumerate}


