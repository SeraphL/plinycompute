% http://www.vldb.org/pvldb/vol9/p1707-pirk.pdf
% https://cs.stanford.edu/~matei/papers/2017/cidr_weld.pdf
% http://www.vldb.org/pvldb/vol7/p853-klonatos.pdf

\section{Introduction}

Big Data systems such as Spark \cite{zaharia2010spark} and Flink \cite{alexandrov2014stratosphere, carbone2015apache}
have effectively solved what we call the ``data munging'' problem.  That is, 
these systems do an excellent job supporting the rapid
and robust development of problem-specific,
distributed/parallel codes that transform a raw dataset into structured 
or semi-structured form.
%, and then
%extract actionable information from the transformed data.
But while existing Big Data systems offer
excellent support for data munging,
there is a class of application for which 
existing systems are 
used, but arguably are far less suitable:
as a platform 
for the development of high-performance codes, especially reusable
Big Data tools and libraries, by a capable
systems programmer.

The desire to build new tools
on top of existing Big Data systems is understandable.  
The  developer of a distributed data processing tool must worry about data persistence, movement of
data to/from secondary storage, data
and task distribution, resource allocation, load balancing, fault tolerance, and many other factors.
While classical high-performance computing (HPC)
tools such as MPI \cite{gropp1996high} do not provide support for all of these concerns,
existing Big Data systems 
address them quite well.
As a result, many tools and libraries have been built on top of existing systems.  For example,
Spark supports
machine learning (ML) libraries Mahout \cite{mahout}, Dl4j \cite{dj4j}, 
and Spark \texttt{mllib} \cite{meng2016mllib}, linear algebra packages such as SystemML \cite{tian2012scalable, boehm2016systemml, ghoting2011systemml, boehm2014hybrid} and Samsara \cite{samsara}, and graph analytics with
GraphX \cite{gonzalez2014graphx} and GraphFrames
\cite{dave2016graphframes}.  Examples abound.

\vspace{5 pt}
\noindent
\textbf{PlinyCompute: A platform for high-performance, Big Data computing.}
However, we assert that if one were to develop a system purely for developing high-performance
Big Data codes
by a capable systems programmer,
it would not look like existing systems such as Spark and Flink,
which have largely 
been built using high-level programming languages and managed runtimes such as the Java 
virtual machine.  Virtual machines abstract away
most details regarding memory management
from the system designer, including memory deallocation, reuse, and movement, as well as pointers,
object serialization and deserialization.
Since managing and utilizing memory is 
one of the most important factors determining big data system performance, reliance
on a managed environment can mean an order-of-magnitude increase in CPU cost for some computations.  
This cost may be acceptable if the person using the system
is a programmer uncomfortable with the basics of memory management who is
building an application to complete a specific data munging task.  
But it is unacceptable for high-performance tool or library
development by an expert. There have been notable efforts to engineer around the limitations of a managed environment and still provide
high performance---Spark's Dataset and
Dataframe abstractions come to mind---but such efforts are necessarily limited compared to
designing a Big Data system from the ground up around special-purpose
memory and object management system.

This paper is concerned with the design and implementation of
\emph{PlinyCompute}, a system for development of
high-performance, data-intensive, distributed computing codes, especially tools and libraries.
PlinyCompute, or PC for short, is designed to fill the gap between
High Performance Computing (HPC) softwares such as OpenMP \cite{dagum1998openmp} and MPI \cite{gropp1996high}, which provide little direct support for
managing very large datasets, and dataflow platforms such as Spark and Flink, which 
may give up significant performance through their reliance on a managed runtime to handle
memory management (including layout and de/allocation) and key computational considerations
such as virtual method/function dispatch to the JVM. 



\vspace{5 pt}
\noindent
\textbf{Core design principle: Declarative in the large, high-perform-
  ance in the small.}
PC is unique in that \emph{in the large}, 
it presents the programmer with a very high-level,
declarative interface, relying on automatic, 
relational-database style optimization \cite{chaudhuri1998overview} to figure out how to stage
distributed computations.  
PC's declarative interface is higher-level than competing systems, in that decisions such as choice of join ordering and which
join algorithms to run are
totally under control of the system. 
This is particularly important for tool and library development because the same tool should run well regardless of the data
it is applied to---the classical idea of \emph{data independence} in database system design \cite{stonebraker1990third}.
A relatively naive library user cannot be expected to tune a library implementation of an algorithm to run
well on his or her particular dataset, and yet with existing systems, this sort of tuning
is absolutely necessary.  For example, we find
that a high quality LDA implementation\footnote{LDA \cite{blei2003latent} is a popular text mining algorithm.}
on top of Spark is around $25\times$ slower than the algorithmically equivalent LDA
implementation on top of PC.  Through careful, dataset-specific tuning (including choosing specific join algorithms and
forcing pipelining of certain results) it is possible to get that gap down to $2.5\times$.  But this requires modification of the
code itself, which is beyond the vast majority of end-users.


In contrast, \emph{in the small}, PlinyCompute presents a capable programmer with a
persistent object data model and API (the ``PC object model'') and associated memory management system
designed from the ground-up for
high performance.
All data processed by PC are managed by
the PC object
model, which is exclusively responsible for PC data layout and within-page memory management.  
The PC object model is tightly coupled with
PC's execution engine, and has been specifically designed for efficient distributed computing.  
All dynamic PC \texttt{Object} allocation is \emph{in-place}, directly on a page, obviating
the need for PC \texttt{Object} serialization and deserialization before data are transferred to/from storage or over a network.
Further, PC gives a programmer fine-grained control of the systems
memory management and PC \texttt{Object} re-use policies.

This hybrid approach---declarative and yet trusting the programmer
to utilize PC's object model effectively
in the small---results in a system that is ideal for the 
development of data-oriented tools and libraries.

Currently, PC exists as a prototype system, consisting of around
150,000 lines of C++ code, with a much smaller amount of Prolog code.
The system consists of three main components: 

\begin{itemize}
\item The \emph{PC object model}, which is a toolkit for building high-performance, persistent data structures that can be 
processed and manipulated by PC.  

\item The \emph{PC API and TCAP compiler}.  In the large, PC codes are declarative and look a lot like classical relational calculus \cite{codd1971data}.  For example, to specify a join over five sets of objects, a PC programmer does not build a join DAG over the five inputs, as in a standard
dataflow platform.  Rather, a programmer 
supplies two \emph{lambda term construction functions}: one that constructs a lambda term describing the selection
predicate over those five input sets, 
and a second that constructs a lambda term describing the relational projection over those five sets
using the same API.  These lambda terms are constructed using PC's built-in lambda abstraction families as well as higher-order composition functions.
 PC's TCAP compiler 
accepts such a specification, and compiles it into a functional, domain specific language called \emph{TCAP} that implements
the join.  Logically, TCAP operates over
sets of columns of PC \texttt{Object}s. 

\item The \emph{PC Cluster}, which is a distributed query processing
  system for big data analytics. It consists of an optimizer for TCAP
  programs, through which all PC computations are optimized using a rule-based
optimizer; we plan to shift to a cost-based optimizer in the
future. Among other components such as a distributed storage that
contains a buffer pool to buffer writes or cache data in pages, it also consists of a
high-performance distributed, vectorized TCAP processing engine. 
The TCAP processing engine is intimately connected to the PC object model, 
and has been designed to work closely with the PC object model to
minimize memory-related costs during computation.



\end{itemize}

\vspace{5 pt}
\noindent
\textbf{Our contributions.}
Taken together, these components allow a competent system programmer
to write exceedingly high-performance distributed codes.
In this paper, we describe the design and implementation of PlinyCompute.  We experimentally show the performance benefits of the PC object model, 
demonstrating how even simple data transformations are much faster using the PC object model compared to similar computations within the 
Apache ecosystem.
In-keeping with PC being targeted at high-performance
library and tool development,
we benchmark several library-style softwares written on top of PC.  We begin with a small domain specific language
for distributed linear algebra that we implemented on top of PC, called \texttt{LilLinAlg}.  \texttt{LilLinAlg} was implemented in about six weeks by a developer
who had no knowledge of PC at the outset, with the goal of demonstrating PC's suitability as a tool-development platform.  
We show that \texttt{LilLinAlg} has better performance than other systems that have long been under development
within the Apache ecosystem.   
We also compare the performance of several standard machine learning codes written on top of PC, comparing them with similar
codes written within the Apache ecosystem.  

