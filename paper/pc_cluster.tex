
\section{Overview of a PC Cluster}

A typical PC cluster configuration is shown in Figure ???.  Each worker machine in a PC cluster runs two processes: a frontend process and a backend process.
Dual processes are used because PC executes potentailly unsafe native user code.  By definition, user code is run only in the backend process---if            
user code happens to crash the backed process, it can be restarted by the front end.

The frontend process runs an instance of Pangaea (providing storage and memory management, including buffering and caching), as well as one more more specialized
servers that handle requests from elsewhere in the PC cluster, as well as from PC users.
The following servers may be running on a front-end process:

\begin{enuermate}

\item A \emph{Catalog server}, serving system meta-data as well as dynamically-loaded code for performing computation over PC \texttt{Object}s (see
Section ??? of the paper);

\item A \emph{Pangaea storage server}, providing access to the process' Pangaea instance;

\item A \emph{Dispatcher server}, responsibly for processing incoming data and distributing it to the Pangaea storage servers in the cluster;

\item A \emph{Distribution manager server}, responsible for monitoring the state of the cluster, including resrouce utilization;

\item A \emph{Compute optimization server}, responsible for optimizing programs that have been compiled into PC's domain-specific TCAP langague; and,

\item A \emph{Compute scheduler server}, responsible accepting optimized TCAP computations, choosing a physical execution plan for the TCAP, and executing the TCAP on the cluster.

\item A  \emph{Compute execution server}, responsible for performing requested computations.

\end{enumerate}
