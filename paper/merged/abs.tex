
\noindent
\textbf{Abstract}

\noindent
This paper describes \emph{PlinyCompute}, a system for development of
high-performance, data-intensive, distributed computing tools and libraries.
\emph{In the large}, PlinyCompute presents the programmer with a very high-level,
declarative interface, relying on automatic, relational-database style optimization to figure out how to stage
distributed computations.  However, \emph{in the small}, PlinyCompute
presents the capable systems programmer with a persistent object data model and API (the ``PC object model'') and associated memory management system
that has been designed from the ground-up for
high performance, distributed, data-intensive computing.
This contrasts with most other Big Data systems, which are constructed on top of the
Java Virtual Machine (JVM), and hence must
at least partially cede performance-critical concerns such as
memory management (including layout and de/allocation) and virtual
method/function dispatch to the JVM.
This hybrid approach---declarative in the large, trusting the programmer's ability
to utilize PC object model efficiently
in the small---results in a system that is ideal for the development of reusable, data-intensive tools and libraries.
Through extensive benchmarking, we show that implementing complex
objects manipulation and non-trivial, library-style computations 
 on top of PlinyCompute can result in a speedup of 2$\times$ to
 more than 50$\times$ or more compared to equivalent implementations on Spark.
