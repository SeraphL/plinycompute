
\section{Conclusions}

This paper has described PlinyCompute, or PC for short.  PC is a system for the development of high performance
distributed data processing tools and libraries.  PC is designed to inhabit the space between
high-performance computing platforms such as OpenMP and MPI, which provide little direct support
for managing very large data sets, and dataflow platforms such as Spark and Flink, which rely on 
a managed runtime to provide low-level services such as allocation and deallocation of data objects and memory management.
PC's guiding design principle is ``declarative in the large,
high performance in the small''.
In the large, PC presents the capable systems programmer with a very high-level, declarative interface, relying on automatic,
relational-database style optimization.  This is crucial for tool and library development, since the same tool should run well
regardless of the characteristics of the data and of the compute platform.  But in the small, PC relies on the PC object model.
The PC object model is an
API for storing and manipulating persistent data, and has been co-designed with PC's memory management
system and computational engine to provide maximum performance.
One of the key ideas behind the PC object model is the \emph{page as a heap principle}. All PC
Objects are allocated and manipulated in-place, on a system- (or user-) allocated page. There is no distinction
between the in-memory representation of data and the on-disk (or in-network) representation of data.
Thus there is no (de-)serialization cost to move data to/from disk and network, and memory management
costs are very low.

We have performed a reasonably extensive set of benchmark experiments that indicate that these ideas can result in 
a system that is very high performance and yet offers a relatively simple and usable object oriented API.  In particular, 
we have given strong evidence that PC \emph{is} particularly well-suited to tool and library development. 
For example, we asked a PhD
student (who at the outset knew nothing of PC) to use the system to build a small Matlab-like programming
language and library for distributed matrix operations called
\texttt{lilLinAlg}.  The resulting tool outperformed many other, long-lived tools
for the same purpose, such as SciDB, \texttt{mllib}, and SystemML.  SystemML in particular has resulted 
in several research papers,
including one awarded a VLDB best paper award \cite{boehm2016systemml}.
\texttt{lilLinAlg} was developed in six weeks by a single PhD student.
One may conjecture that had SystemML been built on a platform such as PC rather than on Spark, it might be significantly
faster than it is now.

Many avenues for future work remain.  In particular, there is significant scope for expanding the functionality and
usability of the PC object model, including relaxing the strict requirement that all handles point within a page.  This
would allow pointer-like objects to point off a page, perhaps even off of a machine.  
