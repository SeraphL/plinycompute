
\section{Optimizing TCAP}
\label{sec:optimizer}

One of the key ideas driving the design and implementation of PlinyCompute is that \emph{all} PC computations should be optimized, both to
match programmer expectation---programmers generally expect
that changes in the way that boolean expressions are composed should not affect system runtime---and 
to protect against poor programmer choices when constructing the query graph.

Optimizability is one of the drivers for
the decision to compile all computations expressed in PC's lambda calculus into TCAP.  
TCAP resembles relational algebra, and it is similarly amenable to rule- and cost-based optimization
using a combination of methods from relational query optimization and classical compiler construction.
Currently, the optimizations implemented in PC are rule-based (such as pushing down selections).  We plan to work on cost-based optimization
in the future---this is a challenging research problem because of a lack of statistics over the data, which are arbitrary PC Objects.

PC's optimizer is currently implemented
in Prolog; a series of transformations are fired iteratively to improve the plan until the plan cannot be improved further.
For an example of the sort of optimization present in PC, consider the task of removing redundant method calls.  Imagine that a user
supplies a \texttt{SelectionComp} with the following \texttt{getSelection ()}:

\begin{codesmall} 
Lambda <bool> getSelection (Handle <Emp> emp) {
        return makeLambdaFromMethod (emp, getSalary) > 50000 &&
		makeLambdaFromMethod (emp, getSalary) < 10000;
}	
\end{codesmall}

\noindent PC would compile this into the following TCAP:

\begin{codesmall}
JK2_1(emp,mt1) <= APPLY(In(emp), In(emp), 'Sel_43', 'method_call_1',
  [('type', 'methodCall'), ('methodName', 'getSalary')]);

JK2_2(emp,bl1) <= APPLY(JK2_1(mt1), JK2_1(emp), 'Sel_43', '>_1', 
  [('type', 'const_comparison'), ('op', '>')]);

JK2_3(emp,bl1,mt2) <= APPLY(JK2_2(emp), JK2_2(emp,bl1), 'Sel_v3', 'method_call_2',
  [('type', 'methodCall'), ('methodName', 'getSalary')]);

JK2_4(emp,bl1,bl2) <= APPLY(JK2_3(mt2), JK2_3(emp,bl1), 'Sel_43', '<_1',
  [('type', 'const comparison'), ('op', '<')]);

JK2_5(emp,bl3) <= APPLY(JK2_4(bl1,bl2), JK2_4(emp), 'Sel_43', '&&_1', 
  [('type', 'bool_and')]);

JK2_6(emp) <= FILTER(JK2_5(bl3), JK2_5(emp), 'Sel_43', []);
\end{codesmall}

\noindent
This TCAP program first calls the method \texttt{getSalary ()} on \texttt{In.emp} to produce a new vector list \texttt{JK2\_2}, storing the result
of the method call in \texttt{JK2\_1.mt1}.  After comparing \texttt{JK2\_2.bl1} to \texttt{50000}, the result of the method call is dropped.
The method is then called once again on \texttt{JK2\_2.emp} and the result compared with \texttt{100000} to produce \texttt{JK2\_4}, at which 
point the two boolean vectors are ``anded'' and the result is filtered.

Obviously, there is a redundancy here as the method \texttt{getSalary ()} will be called twice.
If \texttt{getSalary} simply accesses a data member, the additional call is costless.  But in the general case a method call may run an arbitrary
computation.  Hence, the second call should automatically be removed as being redundant 
(by definition, all method calls evaluated during computation should
be purely functional, and so they must return the same value when called a second time).
The rule TCAP optimization rule leading to its removal is:

\begin{itemize}
\vspace{-5 pt}
\item If two \texttt{APPLY} operations are both of type \texttt{methodCall} and both invoke the same \texttt{methodName};
\vspace{-5 pt}
\item And one \texttt{APPLY} operation is the ancestor of the other in the TCAP graph;
\vspace{-5 pt}
\item And both \texttt{APPLY} operations operate over the same data object;
\vspace{-5 pt}
\item Then the second \texttt{APPLY} operation can be removed, and the result of the first \texttt{APPLY} carried through the graph.
\end{itemize}

\noindent
In our example, the
optimized
TCAP program is:

\begin{codesmall}
JK2_1(emp,mt1) <= APPLY(In(emp), In(emp), 'Sel_43', 'method_call_1',
   [('type', 'methodCall'), ('methodName', 'getSalary')]);

JK2_2(emp,mt1,bl1) <= APPLY(JK2_1(mt1), JK2_1(emp,mt1), 'Sel_43', '>_1', 
  [('type', 'const comparison'), ('op', '>')]);

JK2_4(emp,bl1,bl2) <= APPLY(JK2_3(mt1), JK2_3(emp,bl1), 'Sel_43', '<_1', 
  [('type', 'const comparison'), ('op', '<')]);

JK2_5(emp,bl3) <= APPLY(JK2_4(bl1,bl2), JK2_4(emp), 'Sel_43', '&&_1', 
  [('type', 'bool_and')]);

JK2_6(emp) <= FILTER(JK2_5(bl3), JK2_5(emp), 'Sel_43', []);
  
\end{codesmall}

\noindent
For another example of a rule-based TCAP optimization, 
consider the classical technique of pushing selection predicates past joins.  Imagine that a user supplied the following
\texttt{getSelection ()} for a \texttt{JoinComp} operation:

\begin{codesmall} 
Lambda <bool> getSelection (Handle <Emp> emp, Handle <Emp> sup) {
        return makeLambdaFromMethod (emp, getSalary) > 50000 &&
		(makeLambdaFromMethod (emp, getSupervisor) == 
                 makeLambdaFromMember (sup, name));
}	
\end{codesmall}

\noindent
Since all selection predicates are by default evaluated \emph{after} the join, 
this would be compiled to the following TCAP code:

\begin{codesmall}
JK2_1(sup,mt1) <= APPLY(InSup(sup), InSup(sup), 'Join_42', 'att_access_1', 
  [('type', 'attAccess'), ('attName', 'name')]);

JK2_2(sup,hash1) <= HASH(JK2_1(mt1), JK2_1(sup), 'Join_42', []);

JK2_3(emp,mt2) <= APPLY(InEmp(emp), InEmp(emp), 'Join_42', 'method_call_1', 
  [('type', 'methodCall'), ('methodName', 'getSupervisor')]);

JK2_4(emp,hash2) <= HASH(JK2_3(mt2), JK2_3(emp), 'Join_42', []);

JK2_5(sup,emp) <= JOIN(JK2_2(hash1), JK2_2(sup), 
  JK2_4(hash2), JK2_4(emp), 'Join_42', []);

JK2_6(sup,emp,mt3) <= APPLY(JK2_5(emp), JK2_5(sup,emp), 'Join_42', 'method_call_2',
  [('type', 'methodCall'), ('methodName', 'getSalary')]);

JK2_7(sup,emp,bool1) <= APPLY(JK2_6(mt2), JK2_7(sup,emp), 'Join_42', '>_1', 
  [('type', 'const comparison'), ('op', '>')]);

/* additional code here to check whether getSupervisor == name... 
   result goes into JK2_10.bool2 */

JK2_11(sup,emp,bool3) <= APPLY(JK2_10(bl1,bl2), JK2_10(sup,emp), 'Join_42', '&&_1', 
   [('type', 'bool_and')]);

JK2_12(sup,emp) <= FILTER(JK2_11(bool2), JK2_11(sup,emp), 'Join_42', []);
\end{codesmall}
 
\noindent
This code first uses \texttt{emp.getSupervisor ()} and \texttt{sup.name} to obtain the join keys. These are hashed, and 
a hash join is run (this is the \texttt{JOIN} operation).  After the hash join,
the result of calling \texttt{emp.getSupervisor ()} is compared with
\texttt{sup.name}.  If these two values are equal and the salary exceeds \texttt{50000}, the result tuple is accepted.

Clearly, it should be possible to first filter based off of the salary exceeding \texttt{50000} before the hash join is ever run.  Hence, one of
the rule-based optimizations available to PC is that:

\begin{itemize}

\vspace{-5 pt}
\item If there is a boolean predicate of the form $(b_1 \wedge b_2 \wedge ...)$ that operations
on the result of a join;

\vspace{-5 pt}
\item And some $b_i$ refers to values that depend only on one of the join inputs (in this case, \texttt{emp.getSupervisor ()}
\texttt{>} \texttt{50000};
depends only upon \texttt{emp});

\vspace{-5 pt}
\item Then $b_i$ can be pushed down to that join input, and a new
  \texttt{FILTER} is introduced.
\end{itemize}

\noindent In this case, after the transformation, we would have:

\begin{codesmall}
JK2_1(sup,mt1) <= APPLY(InSup(sup), InSup(sup), 'Join_42', 'att_access_1', 
  [('type', 'attAccess'), ('attName', 'name')]);

JK2_2(sup,hash1) <= HASH(JK2_1(mt1), JK2_1(sup), 'Join_42', []);

JK2_6(emp,mt3) <= APPLY(InEmp(emp), InEmp(emp), 'Join_42', 'method_call_2', 
  [('type', 'methodCall'), ('methodName', 'getSalary')]);

JK2_7(emp,bool1) <= APPLY(JK2_6(mt2), JK2_7(emp), 'Join_42', '>_1', 
  [('type', 'const comparison'), ('op', '>')]);

JK_2_7_1(emp) <= FILTER(JK2_7(bool1), JK2_7(emp), 'Join_42', []);

JK2_3(emp,mt2) <= APPLY(JK_2_7_1(emp), JK_2_7_1(emp), 'Join_42', 'method_call_1', 
  [('type', 'methodCall'), ('methodName', 'getSupervisor')]);

JK2_4(emp,hash2) <= HASH(JK2_3(mt2), JK2_3(emp), 'Join_42', []);

JK2_5(sup,emp) <= JOIN(JK2_2(hash1), JK2_2(sup), JK2_4(hash2), 
  JK2_4(emp), 'Join_42', []);

/* additional code here to check whether getSupervisor == name... 
   result goes into JK2_10.bool2 */

JK2_11(sup,emp,bool3) <= APPLY(JK2_10(bl1,bl2), JK2_10(sup,emp), 'Join_42', '&&_1', 
  [('type', 'bool_and')]);

JK2_12(sup,emp) <= FILTER(JK2_11(bool2), JK2_11(sup,emp), 'Join_42', []);
\end{codesmall}
