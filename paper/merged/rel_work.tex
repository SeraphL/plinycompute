
\section{Related Work}

The PC project was inspired by works from database engines, distributed in-memory processing
frameworks,  programming languages and compilers, which can be
categorized into two classes for convenience of discussions: systems and techniques with managed
runtimes, and without managed runtimes.


\subsection {System and Optimizations with Managed Runtimes}

Most big data query processing and analytics
systems developed in recent are implemented in
high-level programming languages such as Java and Scala  ~\cite{dean2008mapreduce, yu2008dryadlinq,
  neumann2011efficiently, zaharia2012resilient,
  alexandrov2014stratosphere, klonatos2014building,
  crotty2015tupleware, armbrust2015spark}, and rely on a managed
runtime such as JVM for memory, object and virtual function call management. Numerous research
works propose techniques aiming at
reducing the overhead incurred by the managed runtime employed in
those systems, including code generation, GC tuning, off-heap memory and structured objects.


\vspace{5pt}
\texttt{Code Generation.} This technique gained popularity in modern
in-memory query processing systems. The basic idea is that a system implemented in high level language
can generate native query execution code that manually manages
memory and avoid virtual function call overheads. Hyper~\cite{neumann2011efficiently} proposes a code
generation strategy for query compilation to translate relational
algebra to LLVM assembler for execution.  LegoBase~\cite{klonatos2014building} switches the interface
from declarative SQL to high-level language (Scala) and uses a query engine
written in Scala as a code generator to emit specialized and low-level
C code for execution. TupleWare~\cite{crotty2015tupleware} supports
high-level language (any language with an LLVM compiler) interfaces
and aims to
optimize for UDFs by utilizing code
generation technique to integrate UDF code with the framework
code. 
Weld~\cite{palkar2017weld} is a recent system developed in Scala and
Python. It proposes
to have a common runtime for data anlytics libraries by asking library
developers to express their work using a new intermediate
representation (IR) and compiles this IR into multi-threaded code using
LLVM, then application developers can use unified APIs to
call different libraries from Weld. 

\vspace{5pt}
\texttt {Optimized Memory Management.} 
Apache Flink~\cite{alexandrov2014stratosphere} and Apache
Spark~\cite{zaharia2012resilient} are both distributed in-memory
dataflow systems. They both provide high level language interfaces like Java and Scala.
Spark SQL~\cite{armbrust2015spark} is a relational system built on top
of Spark. Those systems all have the
ability to alleviate the garbage collection overhead  by storing data
into byte arrays or even storing data off-heap.

Apache Flink~\cite{alexandrov2014stratosphere} assigns memory budgets to its data processing operators. Upon
initialization, a computation requests its memory budget from the
memory manager and receives a corresponding set of byte arrays as memory segments. So
each operation will have its own memory pool that it can manually
manage. 

Apache SparkSQL~\cite{armbrust2015spark} serializes simple data objects (or
relational table in SparkSQL) into byte arrays and store the serialized bytes
in a main-memory columnar storage. Spark Tungsten~\cite{tungsten}
optimizes Spark execution backend by grouping execution
data such as hash-based aggregation and shuffle data
to byte arrays and data can be allocated in off-heap mode via
Java APIs such as sun.misc.Unsafe to
further reduce GC overhead. Deca~\cite{lu2016lifetime} is a memory management framework aiming at
reducing GC overhead. It stores
various Spark data types, e.g. UDF variables, user data and
shuffle data into different
off-heap containers so that objects in each container can have similar
lifetime and can be recycled together.

\vspace{5pt}
\texttt {Relational Processing on Binary or Structured Objects.} The idea is to convert or
serialize a Java Object into a binary representation and directly
operate on this binary representation. One significant benefit is to reduce serialization and
deserialization overhead, which are proved to be expensive~\cite{ousterhout2015making}. 
Besides that, binary data representation is also more space efficient. In addition, binary objects
with similar lifetimes can be easily stored in consecutive byte
arrays, and applying this to long-living objects can significantly reduce GC overhead.

Apache Flink~\cite{alexandrov2014stratosphere} uses reflection-based
techniques to analyze Java/Scala object type information, and
map each object type to one of a limited set of
fundamental data types, such as Java primitive type, array type,
Hadoop Writable type, Flink fixed-length tuples and so on. Each
fundamental data type is associated with a serializer and certain data
types provides comparators to efficiently compare binary
representations and extract fixed-length binary key prefixes without
deserializing the whole object.

In recent, Spark~\cite{tungsten} introduces Dataset/Dataframe representations
in addition to RDD~\cite{zaharia2012resilient}. Dataset/Dataframe is
in nature binary data representation encoded from JVM objects to
abstract an object  into a row with fields/columns. Internally,
Spark distinguishes DataSet and Dataframe as strongly typed representation and
untyped representation respectively, and a Dataframe can be
converted from a Dataset through $ DataFrame =
Dataset[Row] $. Dataset/Dataframe enables relational-style processing
through a relational query optimizer called Catalyst and
also enables Java intermediate code generation to reduce virtual
function call overhead through Tungsten~\cite{tungsten}. 

However while
such techniques can boost performance significantly, they also move the
systems closer to relational processing, and tend to limit the
expressiveness of data representations and the flexibility of
programming models.


\vspace{5pt}
Compared with these works, 
PC distinguishes itself in following aspects:
(1) it combines a highly declarative object-oriented interface with a
low-level native
C++-based query engine, (2) it builds a unique PC Object Model on top of
native C++ to move objects around a distributed system with no
serialization/deserialization costs, (3) it uses a unique TCAP
compiler to generate optimized, native, and vectorized
pipeline code with thread-local and tunable memory management for executing
user defined queries based on complex objects.

\subsection {Systems without Managed Runtimes}
Despite the decreasing performance gap between Java/Scala and C/C++,
OS and many tools or libraries are developed in C/C++. For example,
a large number of popular numerical processing libraries such as
various BLAS implementations~\cite{barrachina2008evaluation}, Eigen, GSL
and etc, are all implemented in C/C++. 
%Even for the BLAS standard, there are multiple
%implementations such as Intel's MKL, cuBLAS, MAGMA, and openBLAS. 
On one hand, to
develop those libraries or tools, developers prefer to have some ``tunability'' to control critical apsects of performance such as
memory management, because they have the greatest knowledge about the
application's performance and behavior. On the other hand,
developers need declarative interfaces and automatic code generation
to reduce their burdens. PC is designed to strike a balance between
the two ends. 

In this section, we mainly describe existing native big
data systems and explain why they are not suitable as tool/library
development environment.

\vspace{5pt} 
\texttt{High Performance Computing (HPC).} HPC
systems such as OpenMP~\cite{dagum1998openmp}, Cilk~\cite{blumofe1996cilk},  Intel's Array Building
Blocks~\cite{newburn2011intel}, and Threading Building Blocks~\cite{reinders2007intel} stretch from
low-level interfaces like MPI~\cite{gropp1996high}, and create powerful control flow, computation
and communication abstractions for multi-core and distributed
platforms. However while those frameworks provide developers the most
``tunability'', the fundamental problem is that they expect users
or applications to write code for strategies and policies of using
those static abstractions. 

\vspace{5pt} 
\texttt{Query Processing.} VectorWise~\cite{zukowski2012vectorwise}, which was commercialized as the MonetDB/X100 project, makes use of
vectorized query execution to amortize the virtual function call overhead and also to exploit SIMD support
on modern hardware. Voodoo~\cite{pirk2016voodoo} uses a new
declarative algebra as the compilation target for query plans to
enable tuning for hardware specifics like caches, SIMD registers,
GPU and so on. It requires a relational frontend, and serves as an alternative backend for
relational databases like MonetDB and Hyper. While PC TCAP compilation
and processing shares
similarity in
code generation and vectorized processing in those
systems and PC can leverage Voodoo's hardware tuning concept, PC is
different with those relational processing systems in that in PC all query processing
is based on a high performance and high expressive object model.
PC can support efficient
processing of complex objects and user defined
dataflows to facilitate tool or library development with more flexible
programming models.

\vspace{5pt} 
\texttt{Cloud Frameworks.} Tensorflow~\cite{abadi2016tensorflow} is a
distributed computing framework mainly designed for deep learning. It
mainly supports processing of numerical data with a
limited set of types, so for example,  it is
difficult to apply Tensorflow to decision support tasks such as
TPCH or other tasks that require a complex object representation. In
addition, it provides a much lower level interface based on tensors,
variables and sessions, compared with PC's
declarative interface. 
%Tensorflow can serve as an alternative backend of
%PC for developing
%numerical processing and deep learning tools.



