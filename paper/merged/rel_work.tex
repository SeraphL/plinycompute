
\section{Related Work}

The PC project was inspired by works from a broad class of
fields, including database engines, distributed in-memory processing
frameworks,  programming languages and compilers. PC is
mainly aiming at providing domain-experts a convenient platform to build high
performance tools and libraries. PC distinguishes itself with existing
works in following aspects:
(1) it combines a highly declarative object-oriented interface with a
low-level native
C++-based query engine, (2) it builds a unique PC Object Model on top of
native C++ to move objects around a distributed system with no
serialization/deserialization costs, (3) it uses a unique TCAP compiler to generate vectorized
pipeline code with thread-local memory management for executing
user-defined object-oriented queries.

\subsection {In-memory Query Processing Systems}
Vector processing and just-in-time compilation techniques are widely
used in modern in-memory query
processing. VectorWise~\cite{zukowski2012vectorwise}, which was
commercialized as the MonetDB/X100 project, makes use of vectorized
query execution to amortize the virtual function call overhead and
also to exploit SIMD support on modern hardware.
Hyper~\cite{neumann2011efficiently} proposed a different code
generation strategy for query compilation to translate relational
algebra to LLVM assembler for execution.
 LegoBase~\cite{klonatos2014building} switched the interface
from declarative SQL to high-level language (Scala) and uses a query engine
written in Scala as a code generator to emit specialized and low-level
C code for execution. TupleWare~\cite{crotty2015tupleware} also supports
high-level language (any language with an LLVM compiler) interfaces
and aims to
optimize for UDFs by utilizing code
generation technique to integrate UDF code with the framework
code. Voodoo~\cite{pirk2016voodoo} served as an alternative backend for
relational databases like MonetDB and Hyper. It used a new
declarative algebra as the compilation target for query plans to
enable tuning for hardware specifics like caches, SIMD registers,
GPU and so on. Weld~\cite{palkar2017weld} is a recent system proposed
to have a common runtime for data anlytics libraries by asking library
developers to express their work using a new intermediate
representation and asking application developers to use new APIs to
call different libraries from Weld.  

\subsection{Distributed Dataflow Processing Systems}
Over last a few years, distributed dataflow computing gained
popularity for big
data analytics. MapReduce~\cite {dean2008mapreduce} and its opensource
implementation Apache Hadoop~\cite{white2009hadoop} is the simplest and
most fundamental form of this model, while systems like
Dryad/DryadLINQ~\cite{yu2008dryadlinq} generalized the types of operators and
data flows supported. In recent years, due to the requirements for
iterative processing, in-memory systems like
Apache Spark~\cite{zaharia2012resilient} and
Apache Flink~\cite{alexandrov2014stratosphere} are widely deployed and
discussed. While those systems are all object-oriented, they all provide a high-level language
(e.g. Java, Scala and Python) and non-declarative interface and are
based on a layered backend that is built on Java virtual machines.

SparkSQL~\cite{armbrust2015spark} builds a SQL query
interfaces, a Catalyst query optimizer and query compiler called Whole-stage Code Generation
on top of the Spark distributed platform. Different with PC, SparkSQL
is based on relational data and can not be used to compose complex
objects and is not designed for extensive use of user defined functions.





